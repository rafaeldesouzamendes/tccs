%% abtex2-modelo-trabalho-academico.tex, v-1.6.1 laurocesar
%% Copyright 2012-2013 by abnTeX2 group at http://abntex2.googlecode.com/ 
%%
%% This work may be distributed and/or modified under the
%% conditions of the LaTeX Project Public License, either version 1.3
%% of this license or (at your option) any later version.
%% The latest version of this license is in
%%   http://www.latex-project.org/lppl.txt
%% and version 1.3 or later is part of all distributions of LaTeX
%% version 2005/12/01 or later.
%%
%% This work has the LPPL maintenance status `maintained'.
%% 
%% The Current Maintainer of this work is the abnTeX2 team, led
%% by Lauro César Araujo. Further information are available on 
%% http://abntex2.googlecode.com/
%%
%% This work consists of the files abntex2-modelo-trabalho-academico.tex,
%% abntex2-modelo-include-comandos and abntex2-modelo-references.bib
%%

% ------------------------------------------------------------------------
% ------------------------------------------------------------------------
% abnTeX2: Modelo de Trabalho Academico (tese de doutorado, dissertacao de
% mestrado e trabalhos monograficos em geral) em conformidade com 
% ABNT NBR 14724:2011: Informacao e documentacao - Trabalhos academicos -
% Apresentacao
% ------------------------------------------------------------------------
% ------------------------------------------------------------------------

% verso e anverso:
\documentclass[12pt,openright,twoside,a4paper,english,brazil]{abntex2}

% apenas verso:	
% \documentclass[12pt,oneside,a4paper,english,french,spanish,brazil]{abntex2} 


% ---
% PACOTES
% ---

% ---
% Pacotes fundamentais 
% ---
\usepackage{cmap}				% Mapear caracteres especiais no PDF
\usepackage{lmodern}			% Usa a fonte Latin Modern			
\usepackage[T1]{fontenc}		% Selecao de codigos de fonte.
\usepackage[utf8]{inputenc}		% Codificacao do documento (conversão automática dos acentos)
\usepackage{lastpage}			% Usado pela Ficha catalográfica
\usepackage{indentfirst}		% Indenta o primeiro parágrafo de cada seção.
\usepackage{color}				% Controle das cores
\usepackage{graphicx}			% Inclusão de gráficos
% ---
		
% ---
% Pacotes adicionais, usados apenas no âmbito do Modelo Canônico do abnteX2
% ---

\usepackage{lipsum}				% para geração de dummy text

% ---

% ---
% Pacotes de citações
% ---

\usepackage[brazilian,hyperpageref]{backref}	 % Paginas com as citações na bibl
\usepackage[alf]{abntex2cite}	% Citações padrão ABNT

% --- 
% CONFIGURAÇÕES DE PACOTES
% --- 

% ---
% Configurações do pacote backref
% Usado sem a opção hyperpageref de backref
\renewcommand{\backrefpagesname}{Citado na(s) página(s):~}
% Texto padrão antes do número das páginas
\renewcommand{\backref}{}
% Define os textos da citação
\renewcommand*{\backrefalt}[4]{
	\ifcase #1 %
		Nenhuma citação no texto.%
	\or
		Citado na página #2.%
	\else
		Citado #1 vezes nas páginas #2.%
	\fi}%
% ---


% ---
% Informações de dados para CAPA e FOLHA DE ROSTO
% ---

\titulo{Algoritmo de Criação Dinâmica da Ontologia para uma Cloud.}
\autor{Maxwell Gonçalves de Almeida}
\local{Florianópolis, SC}
\data{Novembro de 2013}
\orientador{Prof. Dr. Carlos Becker Westphall}
\coorientador{Prof. Ms. Rafael de Souza Mendes}
\instituicao{%
  Universidade Federal de Santa Catarina
  \par
  Departamento de Informática e Estatística
  \par
  Bacharelado em Ciência da Computação}
\tipotrabalho{Trabalho de Conclusão de Curso}

% O preambulo deve conter o tipo do trabalho, o objetivo, 
% o nome da instituição e a área de concentração 

\preambulo{Trabalho de Conclusão de Curso submetido à Universidade Federal de
Santa Catarina como parte dos requisitos para a obtenção do grau de Bacharel em
Ciência da Computação.}
% ---


% ---
% Configurações de aparência do PDF final

% alterando o aspecto da cor azul
\definecolor{blue}{RGB}{41,5,195}

% informações do PDF
\makeatletter
\hypersetup{
     	%pagebackref=true,
		pdftitle={\@title}, 
		pdfauthor={\@author},
    	pdfsubject={\imprimirpreambulo},
	    pdfcreator={LaTeX with abnTeX2},
		pdfkeywords={abnt}{latex}{abntex}{abntex2}{trabalho acadêmico}, 
		colorlinks=true,       		% false: boxed links; true: colored links
    	linkcolor=blue,          	% color of internal links
    	citecolor=blue,        		% color of links to bibliography
    	filecolor=magenta,      		% color of file links
		urlcolor=blue,
		bookmarksdepth=4
}
\makeatother
% --- 

% --- 
% Espaçamentos entre linhas e parágrafos 
% --- 

% O tamanho do parágrafo é dado por:
\setlength{\parindent}{1.3cm}

% Controle do espaçamento entre um parágrafo e outro:
\setlength{\parskip}{0.2cm}  % tente também \onelineskip

% ---
% compila o indice
% ---

\makeindex
% ---

% -------------------
% ----
% ----
% Início do documento
% ----
% ----
% ----
% -------------------

\begin{document}

% Retira espaço extra obsoleto entre as frases.

\frenchspacing 

% ----------------------------------------------------------
% ELEMENTOS PRÉ-TEXTUAIS
% ----------------------------------------------------------

\pretextual

% ---
% Capa
% ---

\imprimircapa
% ---

% ---
% Folha de rosto
% (o * indica que haverá a ficha bibliográfica)
% ---

\imprimirfolhaderosto*
% ---
% FIXME
% ---
% Inserir a ficha bibliografica
% ---

% Isto é um exemplo de Ficha Catalográfica, ou ``Dados internacionais de
% catalogação-na-publicação''. Você pode utilizar este modelo como referência. 
% Porém, provavelmente a biblioteca da sua universidade lhe fornecerá um PDF
% com a ficha catalográfica definitiva após a defesa do trabalho. Quando estiver
% com o documento, salve-o como PDF no diretório do seu projeto e substitua todo
% o conteúdo de implementação deste arquivo pelo comando abaixo:
%
% \begin{fichacatalografica}
%     \includepdf{fig_ficha_catalografica.pdf}
% \end{fichacatalografica}

\begin{fichacatalografica}
	\vspace*{\fill}					% Posição vertical
	\hrule							% Linha horizontal
	\begin{center}					% Minipage Centralizado
	\begin{minipage}[c]{12.5cm}		% Largura
	
	\imprimirautor
	
	\hspace{0.5cm} \imprimirtitulo  / \imprimirautor. --
	\imprimirlocal, \imprimirdata-
% FIXME	
	\hspace{0.5cm} \pageref{LastPage} p. : il. (algumas color.) ; 30 cm.\\
	
	\hspace{0.5cm} \imprimirorientadorRotulo~\imprimirorientador\\
	
	\hspace{0.5cm}
	\parbox[t]{\textwidth}{\imprimirtipotrabalho~--~\imprimirinstituicao,
	\imprimirdata.}\\
	
	\hspace{0.5cm}
		1. Ontologia.
		2. Computação Autonômica.
		3. Computação em Nuvem.
		4. Meta-dados.
		5. Monitoramento em Rede.
		6. Topologia da Nuvem.
		7. Gerência de Redes.
		I. Prof. Dr. Carlos Becker Westphall.
		II. Universidade Federal de Santa Catarina.
		III. Departamento de Informática e Estatística.
		IV. Bacharel em Ciência da Computação\\ 			
	
	\hspace{8.75cm} CDU 02:141:005.7\\
	
	\end{minipage}
	\end{center}
	\hrule
\end{fichacatalografica}
% ---

% ---
% Inserir errata
% ---

\begin{errata}
% FIXME
\vspace{\onelineskip}

WESTPHALL, C. B. et Al. \textbf{Algoritmo de Criação Dinâmica de uma Ontologia
para Cloud}: algoritmo que instancia dinamicamente uma ontologia para uma
Cloud.2013.11 f. Artigo Ciêntífico - Departamento de Informática e Estatística,
Universidade Federal de Santa Catarina, Florianópolis, 2013.

\begin{table}[htb]
\center
\footnotesize
\begin{tabular}{|p{1.4cm}|p{1cm}|p{3cm}|p{3cm}|}
  \hline
   \textbf{Folha} & \textbf{Linha}  & \textbf{Onde se lê}  & \textbf{Leia-se}  \\
    \hline
    2 & 1 & de nível & em nivel\\
   \hline
\end{tabular}
\end{table}

\end{errata}
% ---

% ---
% Inserir folha de aprovação
% ---

% Isto é um exemplo de Folha de aprovação, elemento obrigatório da NBR
% 14724/2011 (seção 4.2.1.3). Você pode utilizar este modelo até a aprovação
% do trabalho. Após isso, substitua todo o conteúdo deste arquivo por uma
% imagem da página assinada pela banca com o comando abaixo:
%
% \includepdf{folhadeaprovacao_final.pdf}
%
\begin{folhadeaprovacao}

  \begin{center}
    {\ABNTEXchapterfont\large\imprimirautor}

    \vspace*{\fill}\vspace*{\fill}
    {\ABNTEXchapterfont\bfseries\Large\imprimirtitulo}
    \vspace*{\fill}
    
    \hspace{.45\textwidth}
%     \begin{minipage}{.5\textwidth}
%         \imprimirpreambulo
%     \end{minipage}%
    \vspace*{\fill}
 \end{center}
    
   Trabalho de graduação sob o título de \textbf{``Algoritmo de Criação Dinâmica
   de uma Ontologia para Cloud''}, defendido por \textit{Maxwell Gonçalves de
   Almeida} e aprovado em \textrm{07 de Novembro de 2013}, em
   \textrm{Florianópolis, Santa Catarina}, pela \textit{banca examinadora}
   constituída por:

   \assinatura{\textbf{\imprimirorientador} \\ Orientador} 
   \assinatura{\textbf{\imprimircoorientador} \\ Coorientador}
   \assinatura{\textbf{Prof.ª Dr.ª Carla Merkle Westphall} \\ Membro da Banca}
   \assinatura{\textbf{Ms. Guilherme Arthur Gerônimo} \\ Membro da Banca}
      
   \begin{center}
    \vspace*{0.5cm}
    {\large\imprimirlocal}
    \par
    {\large\imprimirdata}
    \vspace*{1cm}
  \end{center}
  
\end{folhadeaprovacao}
% ---

% ---
% Dedicatória
% ---

\begin{dedicatoria}
   \vspace*{\fill}
   \centering
   \noindent
   \textit{ Aos meus e à Deus.} \vspace*{\fill}
\end{dedicatoria}
% ---

% ---
% Agradecimentos
% ---

\begin{agradecimentos}

Os agradecimentos principais são direcionados à todas pessoas do
\emph{Laboratório de Redes e Gerência (LRG)} \footnote{Laboratório de Redes e
Gerência (\textbf{LRG}) \url{http://www.lrg.ufsc.br}} e, a minha esposa, meus
pais, familiares e amigos.
\end{agradecimentos}
% ---

% ---
% Epígrafe
% ---

\begin{epigrafe}
    \vspace*{\fill}
	\begin{flushright}
		\textit{``Não vos amoldeis às estruturas deste mundo, \\
		mas transformai-vos pela renovação da mente, \\
		a fim de distinguir qual é a vontade de Deus: \\
		o que é bom, o que lhe é agradável, o que é perfeito.\\
		(Bíblia Sagrada, Romanos 12, 2)}
	\end{flushright}
\end{epigrafe}
% ---

% ---
% inserir o sumario
% ---

\pdfbookmark[0]{\contentsname}{toc}
\tableofcontents*
\cleardoublepage
% ---

% ---
% inserir lista de ilustrações
% ---

\pdfbookmark[0]{\listfigurename}{lof}
\listoffigures*
\cleardoublepage
% ---

% ---
% inserir lista de tabelas
% ---

\pdfbookmark[0]{\listtablename}{lot}
\listoftables*
\cleardoublepage
% ---

% ---
% inserir lista de abreviaturas e siglas
% ---

\begin{siglas}
  \item[CN] Computação em Nuvem
  \item[DL] Lógicas de Descrição
  \item[EaaS] Everything as a Service
  \item[EU] Expected Utility
  \item[GC] Grid Computing
  \item[GNU] GNU Not Unix
  \item[HaaS] Hardware as a Service
  \item[IP] Internet Protocol
  \item[IT] Information Technology
  \item[IaaS] Information as a Service
  \item[MIB] Management Informations Base
  \item[NIST] National Institute Standards and Technology
  \item[OWL-S] Web Ontology Language for Web Services
  \item[PaaS] Platform as a Service
  \item[PCMONS] Private Cloud Monitoring System
  \item[QoS] Quality of Service
  \item[RDF] Resource Description Framework
  \item[SaaS] Software as a Service 
  \item[SLO] Service Layer Objects
  \item[SMI] Structure of Management Information
  \item[SNMP] Standard Network Management Framework
  \item[SOA] Service-Oriented Architecture
  \item[RPC] Remote Procedure Call
  \item[UC] Utility Computing
  \item[VM] Virtual Machine
  \item[W3C] World Wide Web Consortium
  \item[XML] eXtensive Markup Language  
\end{siglas}
% ---

% ---
% inserir lista de símbolos
% ---

\begin{simbolos}
  \item[$ \Gamma $] Letra grega Gama
  \item[$ \Lambda $] Lambda
  \item[$ \zeta $] Letra grega minúscula zeta
  \item[$ \in $] Pertence
\end{simbolos}
% ---

% ---
% RESUMOS
% ---

% % resumo em português
%  Segundo a \citeonline[3.1-3.2]{NBR6028:2003}, o resumo deve ressaltar o
%  objetivo, o método, os resultados e as conclusões do documento. A ordem e a extensão
%  destes itens dependem do tipo de resumo (informativo ou indicativo) e do
%  tratamento que cada item recebe no documento original. O resumo deve ser
%  precedido da referência do documento, com exceção do resumo inserido no
%  próprio documento. (\ldots) As palavras-chave devem figurar logo abaixo do
%  resumo, antecedidas da expressão Palavras-chave:, separadas entre si por
%  ponto e finalizadas também por ponto.

\begin{resumo}

Análise comparada das principais ferramentas de levantamento de topologia de
rede, elementos e meta-dados por meio da MIB SNMP. Seleção de meta-dados
relevantes nessas ferramentas e implementação do algoritmo de criação dinâmica
da ontologia para uma Cloud.
 \vspace{\onelineskip}
    
 \noindent
 \textbf{Palavras-chaves}: ontologia. meta-dados. computação autonômica.
 computação em nuvem. topologia em cloud. gerência em redes.
\end{resumo}

% resumo em inglês
\begin{resumo}[Abstract]
 \begin{otherlanguage*}{english}
 
Comparative analysis of the main tools for obtaining the topology network,
elements and metadata through the SNMP MIB. Selection of metadata in these
relevant tools and algorithm implementation of dynamic creation ontology for a
Cloud.

   \vspace{\onelineskip}
 
   \noindent 
   \textbf{Key-words}: ontology. meta-data. autonomic computing. cloud
   computing. cloud topology. network management.
 \end{otherlanguage*}
\end{resumo}
% ---



% ----------------------------------------------------------
% ELEMENTOS TEXTUAIS
% ----------------------------------------------------------

\textual



% ----------------------------------------------------------
% PARTE - Introdução
% ----------------------------------------------------------

\part{Introdução}

% ----------------------------------------------------------
% Capítulo de Introdução
% ----------------------------------------------------------
\chapter*[Introdução]{Introdução}
\addcontentsline{toc}{chapter}{Introdução}

O nome \textsf{nuvem} aplicado à \textit{Infomation Techonology (IT)}, é segundo
\citeonline{Velte2009} uma metáfora para a forma como a Internet é normalmente
indicada nos diagramas de topologia de rede, representando todas as tecnologias
que a fazem funcionar e, também, abstraindo a infraestrutura ou outras
complexidades envolvidas. A \textsf{Computação em Nuvem (CN)} é uma nova forma
de fornecer recursos computacionais facilmente e com transparência através da
Internet que já está mais que difundida e, diversos nichos de mercado vêem a
nuvem como uma alternativa para a implantação dos seus serviços. Segundo a
definição do \textit{National Institute Standards and Technology (NIST)}, ``é um
modelo para permitir acesso \textit{on-demand} de rede a um \textit{pool}
compartilhado de recursos computacionais [\ldots] que podem ser rapidamente
provisionados e removidos com um esforço mínimo de gerenciamento e interação dos
provedores de serviços.'', \citeonline[p.~6]{Mell2009}.

Em termos práticos a CN promete reduzir os custos operacionais e de capital e,
mais importante, deixar os departamentos de IT se concentrarem em projetos
estratégicos, ao invés de manter um \textit{datacenter} em execução, portanto, a
principal vantagem da nuvem em relação ao datacenter tradicional é a capacidade
de expandir seus recursos e otimizar a sua utilização, característica conhecida
como \emph{elasticidade}, \apudonline[p.~1]{Schubert2013}{Mell2009}. Esta
elasticidade possibilita o usuário da nuvem obter recursos e reciclar estes
recursos quando não mais utilizados, pagando apenas pelo período em que
efetivamente os utilizou:

\begin{citacao}
A nuvem é o estágio atual da evolução da Internet, que fornece os meios através
do qual tudo é entregue como um serviço, onde e quando for preciso. Pode
acessá-los em casa, tê-los hospedado, terceirizá-los inteiramente, ou
adquiri-los através da nuvem. No final a maioria das organizações terá um
ambiente híbrido compreendendo serviços a partir de múltiplas fontes. Isso não
significa que \textbf{nem todos os processos de negócios com a tecnologia serão
movidos para a nuvem}, longe disso. As empresas vão querer dar uma olhada em
seus processos mais estratégicos de negócio, propriedade intelectual e
informações de negócios, e determinar quais ativos de computação deve continuar
a ser entregues por meio de modelos de entrega de tecnologia tradicional e quais
estão maduras para tirar proveito dos recursos oferecidos pelo a nuvem.,
\cite[p.~3, grifo nosso]{Hurwitz2009}.
\end{citacao}

Como escreveu \citeonline[p.~1]{Vouk2008}, para chegar ao atual desenvolvimento
da tecnologia foi necessário diversos anos de \textit{Pesquisa \&
Desenvolvimento (P\&D)} onde cabe destacar os ramos da \textit{Utility Computing
(UC)}, \textit{Grid Computing (GC))}, Virtualization, \textit{Cluster Computing
(CC)}, ou seja, em uma definição mais alargada: \emph{Computação Distribuida}. A
simples combinação de tecnologias não viabiliza a implementação deste novo
paradigma, que dentre outras coisas, necessita de mais desenvolvimento nas
ferramentas de \emph{monitoramento e serviços}, também, pesquisas em
\emph{segurança} com a finalidade de detectar este tipo de problema, bem como,
fornecer uma maneira para os administradores de rede definirem e avaliarem
\emph{métricas de segurança}.

Portanto, há desafios de segurança e gerência que ainda precisam ser tratados
garantindo o sucesso da utilização da nuvem. Por exemplo, numerosas ameaças e
vulnerabilidades tornam-se mais relevantes quando a utilização da nuvem aumenta,
assim como, a preocupação com os dados armazenados e sua:
\textsf{disponibilidade}, \textsf{confidencialidade} e \textsf{integridade}. O
que gera expectativas e preocupações por parte de todos tipos de utilizadores da
nuvem: individuais, organizações governamentais ou comerciais.


Entre essas preocupações, segurança e privacidade são as maiores [2]. Isto vem
do fato de que os dados que pertencem a usuários e organizações podem não estar
mais sob seu controle absoluto, sendo agora armazenados em locais de terceiros e
sujeito às suas políticas de segurança, no caso de nuvens públicas. Mas mesmo em
nuvens privadas, o caso mais comum em empresas de telecomunicações, há novos
desafios de segurança, tais como o fornecimento de acesso a um número cada vez
maior de usuários, mantendo o controle de acesso eficiente e bem monitorado.
Torna-se necessário caracterizar o que são os novos riscos associados com a
nuvem e que outros riscos se tornam mais críticos. Estes riscos devem ser
avaliados e mitigados antes da transição para a nuvem.

Já é possível encontrar na literatura uma grande quantidade de trabalho que está
sendo feito nos aspectos de segurança de Cloud Computing, descrevendo seus
desafios e vulnerabilidades e até mesmo propondo algumas soluções [3]. Fornecer
alguma experiência em questões de segurança na computação em nuvem, descrever
brevemente uma implementação anterior de uma ferramenta de monitoramento para a
nuvem, mostrar como as informações de segurança pode ser resumido e tratada sob
uma perspectiva de gerenciamento em um Acordo de Nível de Serviço (SLA ) e, em
seguida, propor um sistema de monitoramento de segurança da informação na nuvem.



\footnote{\url{http://www.latex-project.org/lppl.txt}}.


% ----------------------------------------------------------
% PARTE - preparação da pesquisa
% ----------------------------------------------------------

\part{Preparação da pesquisa}

% ---
% Capitulo de preparação da pesquisa
% ---

\chapter{Preparação da Pesquisa}

% ---
\section{Motivação}
% ---

Com a utilização de computação autonômica para gerar uma ontologia de uma base
de conhecimentos para uma  Cloud, damos um grande passo para implementação de
uma nuvem privada seguinda uma arquitetura específica.

\section{Objetivo Geral}
\label{introducao:objetivos}
\begin{itemize}
\setlength{\itemsep}{1pt}
\setlength{\parskip}{0pt}
\setlength{\parsep}{0pt}
\item Análise comparada das principais ferramentas de levantamento de topologia de
rede, elementos e meta-dados por meio da MIB SNMP.
\item Seleção de meta-dados relevantes nessas ferramentas e implementação do
 algoritmo de criação dinâmica da ontologia para uma Cloud.
\end{itemize}

\section{Objetivos Específicos}
\label{introducao:objetivosespecificos}
\begin{itemize}
\setlength{\itemsep}{1pt}
\setlength{\parskip}{0pt}
\setlength{\parsep}{0pt}
\item Apresentar os conceitos de tecnologias envolvidas na computação em nuvem;
\item Pesquisar ferramentas para implantação e levantamento da topologia de um
ambiente de computação em nuvem e compará-los,
\item Desenvolver um algoritmo de criação dinâmica de ontologia para a nuvem,
\item Testar o algoritmo desenvolvido através de um estudo de caso.
\end{itemize}

\section{Organização do Trabalho} 
\label{introducao:organizacao}

\begin{description}

\item[Capítulo 1 – Introdução -] Apresenta introdução e contextualização ao tema.  
\item[Capítulo 2 – Preparação da Pesquisa -] Apresenta a motivação, objetivo
geral e os objetivos específicos do trabalho.
\item[Capítulo 3 – Revisão da literatura -] Apresenta a definição das
tecnologias envolvidas, definição de ontologia, algoritmos criação dinâmica, são
apresentadas algumas tentativas de padronizações utilizadas. São mostradas e
comparadas ferramentas para o levantamento de meta-dados na MIB.
\item[Capítulo 4 – Resultados -] Nesse capítulo são apresentados os passos para
a implantação de uma nuvem. As melhores opções de ferramentas
para levantamento da MIB. Definição da ontologia e apresentação algoritmo
desenvolvido para criação da ontologia. Bem como, os resultados obtidos após a
criação da mesma.
\item[Capítulo 5 – Conclusão e Trabalhos Futuros -] Este capítulo encerra o
trabalho com algumas conclusões e considerações finais. São apresentadas algumas
perspectivas para trabalhos futuros.
\end{description}

% ----------------------------------------------------------
% Parte de revisão de literatura
% ----------------------------------------------------------

\part{Revisão de Literatura}

% ---
% Capitulo de revisão de literatura
% ---

\chapter{Lorem ipsum dolor sit amet}

% ---
\section{Aliquam vestibulum fringilla lorem}
% ---

\lipsum[1]

\lipsum[2-3]
% ----------------------------------------------------------
% Resultados
% ----------------------------------------------------------

\part{Resultados}

% ---
% Capitulo de Resultados - 1
% ---

\chapter{Lectus lobortis condimentum}

% ---
\section{Vestibulum ante ipsum primis in faucibus orci luctus et ultrices
posuere cubilia Curae}
% ---

\lipsum[21-22]

% ---
% Capitulo de Resultados - 2
% ---

\chapter{Nam sed tellus sit amet lectus urna ullamcorper tristique interdum
elementum}

% ---
\section{Pellentesque sit amet pede ac sem eleifend consectetuer}
% ---

\lipsum[24]



% ---
% Finaliza a parte no bookmark do PDF, para que se inicie o bookmark na raiz
% ---

\bookmarksetup{startatroot}% 
% ---



% ---
% Conclusão
% ---

\chapter*[Conclusão]{Conclusão}

\addcontentsline{toc}{chapter}{Conclusão}

\lipsum[31-33]


% ----------------------------------------------------------
% ELEMENTOS PÓS-TEXTUAIS
% ----------------------------------------------------------

\postextual


% ----------------------------------------------------------
% Referências bibliográficas
% ----------------------------------------------------------

\bibliography{referencias}

% ----------------------------------------------------------
% Glossário
% ----------------------------------------------------------
%
% Consulte o manual da classe abntex2 para orientações sobre o glossário.
%
%\glossary

% ----------------------------------------------------------
% Apêndices
% ----------------------------------------------------------

% ---
% Inicia os apêndices
% ---

\begin{apendicesenv}

% Imprime uma página indicando o início dos apêndices

\partapendices

% ----------------------------------------------------------
\chapter{Quisque libero justo}
% ----------------------------------------------------------

\lipsum[50]

% ----------------------------------------------------------
\chapter{Nullam elementum urna vel imperdiet sodales elit ipsum pharetra ligula
ac pretium ante justo a nulla curabitur tristique arcu eu metus}
% ----------------------------------------------------------
\lipsum[55-57]

\end{apendicesenv}
% ---
% ----------------------------------------------------------
% Anexos
% ----------------------------------------------------------

% ---
% Inicia os anexos
% ---
\begin{anexosenv}

% Imprime uma página indicando o início dos anexos

\partanexos

% ---
\chapter{Morbi ultrices rutrum lorem.}
% ---
\lipsum[30]

% ---
\chapter{Cras non urna sed feugiat cum sociis natoque penatibus et magnis dis
parturient montes nascetur ridiculus mus}
% ---

\lipsum[31]

% ---
\chapter{Fusce facilisis lacinia dui}
% ---

\lipsum[32]

\end{anexosenv}

%---------------------------------------------------------------------
% INDICE REMISSIVO
%---------------------------------------------------------------------

\printindex

\end{document}
