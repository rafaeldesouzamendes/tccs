% ----------------------------------------------------------
% PARTE - Introdução
% ----------------------------------------------------------

\part{Introdução}

% ----------------------------------------------------------
% Capítulo de Introdução
% ----------------------------------------------------------
\chapter*[Introdução]{Introdução}
%\addcontentsline{toc}{chapter}{Introdução}

O nome \textsf{nuvem} aplicado à \textsf{Infomation Techonology (IT)}, é segundo
\citeonline{Velte2009} uma metáfora para a forma como a Internet é normalmente
indicada nos diagramas de topologia de rede, representando todas as tecnologias
que a fazem funcionar e, também, abstraindo a infraestrutura ou outras
complexidades envolvidas. A \textsf{Computação em Nuvem (CN)} é uma nova forma
de fornecer recursos computacionais facilmente e com transparência através da
Internet que já está mais que difundida e, diversos nichos de mercado vêem a
nuvem como uma alternativa para a implantação dos seus serviços. Segundo a
definição do \textsf{National Institute Standards and Technology (NIST)}, ``é um
modelo para permitir acesso \textit{on-demand} de rede a um \textit{pool}
compartilhado de recursos computacionais [\ldots] que podem ser rapidamente
provisionados e removidos com um esforço mínimo de gerenciamento e interação dos
provedores de serviços.'', \citeonline[p.~6]{Mell2009}.

Em termos práticos a CN promete reduzir os custos operacionais e de capital e,
mais importante, deixar os departamentos de IT se concentrarem em projetos
estratégicos, ao invés de manter um \textit{datacenter} em execução, portanto, a
principal vantagem da nuvem em relação ao datacenter tradicional é a capacidade
de expandir seus recursos e otimizar a sua utilização, característica conhecida
como \emph{elasticidade}, \apudonline[p.~1]{Schubert2013}{Mell2009}. Esta
elasticidade possibilita o usuário da nuvem obter recursos e reciclar estes
recursos quando não mais utilizados, pagando apenas pelo período em que
efetivamente os utilizou:

\begin{citacao}
A nuvem é o estágio atual da evolução da Internet, que fornece os meios através
do qual tudo é entregue como um serviço, onde e quando for preciso. Pode
acessá-los em casa, tê-los hospedado, terceirizá-los inteiramente, ou
adquiri-los através da nuvem. No final a maioria das organizações terá um
ambiente híbrido compreendendo serviços a partir de múltiplas fontes. Isso não
significa que \textbf{nem todos os processos de negócios com a tecnologia serão
movidos para a nuvem}, longe disso. As empresas vão querer dar uma olhada em
seus processos mais estratégicos de negócio, propriedade intelectual e
informações de negócios, e determinar quais ativos de computação deve continuar
a ser entregues por meio de modelos de entrega de tecnologia tradicional e quais
estão maduras para tirar proveito dos recursos oferecidos pelo a nuvem.,
\cite[p.~3, grifo nosso]{Hurwitz2009}.
\end{citacao}

Como escreveu \citeonline[p.~1]{Vouk2008}, para chegar ao atual desenvolvimento
da tecnologia foi necessário diversos anos de \textit{Pesquisa \&
Desenvolvimento (P\&D)} onde cabe destacar os ramos da \textsf{Utility Computing
(UC)}, \textsf{Grid Computing (GC))}, Virtualization, \textsf{Cluster Computing
(CC)}, ou seja, em uma definição mais alargada: \emph{Computação Distribuida}. A
simples combinação de tecnologias não viabiliza a implementação deste novo
paradigma, que dentre outras coisas, necessita de mais desenvolvimento nas
ferramentas de \emph{monitoramento e serviços}, também, pesquisas em
\emph{segurança} com a finalidade de detectar este tipo de problema, bem como,
fornecer uma maneira para os administradores de rede definirem e avaliarem
\emph{métricas de segurança}.

Portanto, há desafios de segurança e gerência que ainda precisam ser tratados
garantindo o sucesso da utilização da nuvem. Por exemplo, numerosas ameaças e
vulnerabilidades tornam-se mais relevantes quando a utilização da nuvem aumenta,
assim como, a preocupação com os dados armazenados e sua:
\textsf{disponibilidade}, \textsf{confidencialidade} e \textsf{integridade}. O
que gera expectativas e preocupações por parte de todos tipos de utilizadores da
nuvem: individuais, organizações governamentais ou comerciais.


Entre essas preocupações, segurança e privacidade são as
maiores, \citeonline[p.~2]{Schubert2013}. Isto vem do fato de que os dados que
pertencem a usuários e organizações podem não estar mais sob seu controle
absoluto, sendo agora armazenados em locais de terceiros e sujeito às suas
políticas de segurança, no caso de nuvens públicas. Mas mesmo em nuvens
privadas, o caso mais comum em empresas de telecomunicações, há novos desafios
de segurança, tais como o fornecimento de acesso a um número cada vez maior de
usuários, mantendo o controle de acesso eficiente e bem monitorado.
Torna-se necessário caracterizar o que são os novos riscos associados com a
nuvem e que outros riscos se tornam mais críticos. Estes riscos devem ser
avaliados e mitigados antes da transição para a nuvem.

Já é possível encontrar na literatura uma grande quantidade de trabalho que está
sendo feito nos aspectos de segurança de Cloud Computing, descrevendo seus
desafios e vulnerabilidades e até mesmo propondo algumas soluções. Fornecer
alguma experiência em questões de segurança na computação em nuvem, descrever
brevemente uma implementação anterior de uma ferramenta de monitoramento para a
nuvem, mostrar como as informações de segurança pode ser resumido e tratada sob
uma perspectiva de gerenciamento em um \textsf{Acordo de Nível de Serviço (SLA)}
e, em seguida, propor um sistema de monitoramento de segurança da informação na
nuvem, \citeonline[p.~2]{Schubert2013}.



%\footnote{\url{http://www.latex-project.org/lppl.txt}}.

