% ----------------------------------------------------------
% Parte de revisão de literatura
% ----------------------------------------------------------

\part{Revisão de Literatura}

% ---
% Capitulo de revisão de literatura
% ---

\chapter[Ontologias]{Ontologias}
%\addcontentsline{toc}{chapter}{Ontologias}

% ---
%\section{Ontologias}
% ---

Ontologia em ciência da computação tem especificação explícita para o domínio de
aplicação [ou de interesse] e permite uma relação dos termos estruturados em um
vocabulário, tido como base – conceito, propriedades, relações, funções e
regras. As Ontologias têm encontrado inúmeras aplicações por meio da Web
semântica, e correlaciona sistemas de base de dados em uma cloud que é uma
plataforma orientada a serviços baseada em SOA (Service-Oriented Architecture).
Os serviços que executam tais atividades são descritos como serviços Web
semântico  na linguagem de ontologias OWL-S (Web Ontology Language for Web
Services),\citeonline[p.~10]{Batista2012}. As pesquisas em ontologias têm se
formalizado através de Lógicas de Descrição (DL), que são fragmentos
(decidíveis) da lógica de primeira ordem. Uma ontologia em DL é dividida em duas
componentes, a TBox, para expressar conceitos e suas inter-relações, e a ABox,
que contém afirmações que sobre instâncias
específicas,\citeonline[p.~3]{Wassermann2013}. A evolução de uma ontologia se dá
tanto na sua informação de domínio que pode ser corrigida ou refinada, como em
resposta a uma mudança no domínio subjacente posto que não é estática as
informações, e portanto, são modificados e/ou aperfeiçoados.

As ontologias empregadas pela Cloud são descritas em linguagem OWL (Web Ontology
Language), que é a linguagem de marcação semântica proposta pelo W3C (World Wide
Web Consortium) como padrão pela Web Semântica para descrever ontologias.
Fundamentada em RDF (Resource Description Framework), OWL permite que usuários
descrevam o vocabulário de um dado domínio em termos de classes, instâncias
(indivíduos), propriedades e relacionamentos,\citeonline[p.~12]{Batista2012}.
Entre as potenciais ferramentas que a priori parecem mais adequadas para a
implementação de dinâmica de ontologias encontra-se o modelo de revisão de
protótipos. Os itens e conceitos são baseados em uma representação protótipo da
categoria. De acordo com a teoria do protótipo, certas características de uma
categoria têm o mesmo estatuto e, assim, exemplos que apresentam todas ou a
maioria dessas características se tornam o protótipo para essa categoria. Itens
que não compartilham a maioria destas características podem ainda pertencer a
essa categoria, mas não representam o protótipo. As relações nas ontologias
podem ser pensadas como as características dos protótipos, e os elementos podem
ou não pertecerem à classe dos protótipos, conforme as propriedades que possuam.
Lembra-se no entanto, que características são mutáveis, o que faz com que as
classificações não sejam fixas.

\lipsum[1]

\lipsum[2-3]

\chapter[SNMP]{SNMP}
%\addcontentsline{toc}{chapter}{SNMP}

% ---
%\section{SNMP}
% ---

As ferramentas de consulta de informação serão neste projeto, as SNMP,
denominadas em sua extensão de Standard Network Management Framework. É muito
utilizado em sistema de gerência, por ser menos complicado, e por isso
difundiu-se muito rápido no meio tecnológico. O seu sistema permite inserir
modificações e informações sem maiores impactos. Ou seja, a solução básica de
gerência (instrumentação) e o protocolo SNMP são muito simples. A versao
Framework é baseada em três documentos:

a) \textsf{Structure of Management Information (SMI)}, com linguagem especifica
para informação gerenciada; 
b) \textsf{Management Informations Base (MIB)}, aqui representando ser a
principal ferramenta de inserção de informações, posto que define as variáveis
de gerência; 
c) \textsf{Simple Network Management Protocol (SNMP)}: está é um protocolo usado
entre gerente e agente para a gerência, em um fluxo de valores de variáveis.

Então, a arquitetura para uma solução SNMP é tida como um modelo fetch-store de
variáveis de gerência mantida por agentes de forma muito simples, mas com
resultados poderosos por meio de ações especiais com efeitos colaterais de
operações store (por exemplo: link up, lind down).

Possui três versões SNMPv1; SNMPv2; SNMPv3. A primeira versão SNMPv1 é uma
versão primitiva o GET obtém um valor de uma variável; já o GET –NEXT permite
caminhar nas variáveis por tabelas de tamanho, mesmo quando não se sabe as quais
são essas variáveis. Nela também se encontra o SET, ferramenta que permite
alterar o valor de uma variável; e a TRAP, que informa os eventos
extraordinários (de agente para gerente). Portanto, o modelo é básico
(Trap-Directed Polling) onde o SNMP se insere na pilha TCP/IP.

\lipsum[4]

\lipsum[5-6]

\chapter[Computação em Nuvem]{Computação em Nuvem}
\label{computacaoemnuvem}

O paradigma de computação em nuvem, por estar envolvido com diversas outras
tecnologias computacionais, muitas vezes tem a sua definição incorporada pelas
definições já consolidadas dessas tecnologias, como a computação em grade, a
computação distribuída, a virtualização, entre outras. Assim, uma definição
padrão aceita universalmente para a computação em nuvem se torna difícil. O
\textsf{NIST}, do governo dos Estados Unidos define computação em nuvem como um
modelo que possibilita acesso conveniente e sob demanda, através da rede, a um
conjunto compartilhado de recursos computacionais configuráveis (rede,
servidores, armazenamento, aplicações e serviços). Esses recursos podem ser
providos rapidamente e liberados com um mínimo de esforço de gerenciamento ou
interação com o provedor do serviço.

Uma definição mais simples do termo seria o provimento de recursos
computacionais para um cliente a partir de uma demanda. O fornecedor desses
recursos abstrai as tecnologias envolvidas e a procedência do recurso para os
usuários finais. Desse modo, um cliente, ao contratar um serviço, se atém apenas
à utilização do recurso em si, abstraindo-se da tecnologia envolvida para o
recebimento desses recursos e até mesmo a origem ou localização geográfica do
recurso. Com isso, a obtenção de recursos, como servidores para armazenamento de
arquivos, servidores web, etc., acaba tornando-se desnecessária, evitando na
maioria das vezes, uma subutilização de recursos computacionais. Uma empresa, ao
comprar a quantidade certa de recursos de TI sob demanda, pode evitar a compra
de equipamentos desnecessários. Bancos de dados, redes, aplicativos, plataformas
e até infraestruturas completas são alguns exemplos de recursos computacionais
que podem ser fornecidos.

\section{Classificações para Computação em Nuvem}
\label{computacaoemnuvem:classificacoes}

Os ambientes de computação em nuvem podem ser classificados seguindo mais de um
critério. A seguir são listadas duas das possíveis classificações para esses
ambientes em relação ao seu modelo de serviços e ao seu modelo de implantação.

\subsection{Software como Serviço (SaaS)} 
\label{computacaoemnuvem:classificacoes:servicos:SaaS}

É um modelo de distribuição de \textit{software} em que os aplicativos são
hospedados por um provedor ou um fornecedor de serviço e disponibilizados aos
clientes, através de uma rede, geralmente a Internet.

Algumas das vantagens da utilização do modelo SaaS são:

\begin{itemize}
\setlength{\itemsep}{1pt}
\setlength{\parskip}{0pt}
\setlength{\parsep}{0pt}
\item Facilidade de administração;
\item Compatibilidade: todos os clientes utilizarão a mesma versão do
aplicativo;
\item Atualizações automáticas, sem necessidade de envolvimento do usuário;
\item Facilidade para criação de ambientes colaborativos;
\item Acessibilidade global.
\end{itemize}
     
Pode-se citar como exemplos: o Google\footnote{http://www.google.com}, com a sua
suíte de aplicativos Google Apps\footnote{http://www.google.com/apps}, incluindo
o servidor de e-mails Gmail\footnote{http://mail.google.com}, e seu conjunto de
aplicativos para escritório Google Docs\footnote{http://docs.google.com}, entre
outros serviços.

\subsection{Plataforma como Serviço (PaaS)} 
\label{computacaoemnuvem:classificacoes:servicos:PaaS}

Nesse modelo, um provedor oferece além de uma infraestrutura, um conjunto de
soluções e ferramentas que um desenvolvedor necessita para criar uma aplicação.
Esse modelo oferece a capacidade de gerenciamento de todas as fases do
desenvolvimento de uma aplicação, desde a modelagem e o planejamento, até a
construção, implantação para testes e manutenção. Esse modelo é uma consequência
direta do modelo de Software como Serviço, descrito na seção
\ref{computacaoemnuvem:classificacoes:servicos:SaaS}. Pode-se citar, como
vantagens da utilização desse modelo:

\begin{itemize}
\setlength{\itemsep}{1pt}
\setlength{\parskip}{0pt}
\setlength{\parsep}{0pt}
\item Não há necessidade de comprar todo o sistema, aplicativos,
plataformas, e ferramentas necessárias para construir, executar e implantar o
aplicativo;
\item Recursos do sistema operacional podem ser alterados e
atualizados com frequência;
\item Equipes de desenvolvimento, geograficamente distribuídas,
podem trabalhar juntas em projetos de desenvolvimento;
\item Despesas gerais minimizadas pela unificação dos esforços no
desenvolvimento.
\end{itemize}

Uma das desvantagens de utilização desse modelo é a confiança na segurança dos
dados que o desenvolvedor ou a empresa devem ter na provedora do serviço. A
ideia de que dados pessoais ou corporativos privados, críticos ou não, não estão
armazenados em domínios próprios pode não ser tolerado por algumas pessoas e
organizações.

Pode-se citar como exemplos de provedores de PaaS: o Google App
Engine\footnote{http://appengine.google.com}, que permite o desenvolvimento de
aplicações nas linguagens de programação Python e Java, o
Heroku\footnote{http://www.heroku.com}, da empresa SalesForce.com que suporta
múltiplas linguagens como Ruby, Node.js, Clojure, Java, Python e Scala e o Cloud
Foundry\footnote{http://www.cloudfoundry.com}, ferramenta de código aberto da
empresa VMWare que fornece uma plataforma para desenvolvimento utilizando
diversas linguagens.

\subsection{Infraestrutura como Serviço (IaaS)} 
\label{computacaoemnuvem:classificacoes:servicos:IaaS}

Nessa modalidade é oferecida ao cliente uma infraestrutura de hardware completa.
Armazenamento, hardware (memória, processamento, etc.), servidores e componentes
de rede são oferecidos aos usuários. O consumidor é capaz de implantar e
executar aplicativos arbitrários, que podem incluir sistemas operacionais
completos e tecnologias de virtualização para o gerenciamento dos recursos
\textsf{NIST}. O cliente não administra ou controla a infraestrutura da
nuvem, porém tem controle sobre o sistemas operacionais, armazenamento, uso de
aplicativos e controle dos componentes de rede.

Esse modelo, referido por alguns autores como \textit{Hardware as a Service
(HaaS)} possui inúmeras vantagens, dentre elas:

\begin{itemize}
\setlength{\itemsep}{1pt}
\setlength{\parskip}{0pt}
\setlength{\parsep}{0pt}
\item Permite que uma empresa mantenha o foco nos produtos e
serviços que oferece, delegando o gerenciamento de tecnologia da organização à
provedora de infraestrutura;
\item A infraestrutura pode ser rapidamente ampliada em caso de
maior demanda e reduzida caso a demanda recue;
\item Redução de custos com utilização ótima de recursos. Uso dos
serviços com base na exigência e somente enquanto forem necessários;
\item Flexibilidade. Permite o acesso a infraestrutura a partir de
qualquer lugar, utilizando diversos dispositivos;
\item Permite economia de energia, diminuindo o impacto ambiental,
contribuindo para iniciativas da chamada TI Verde ou \textit{Green IT}.
\end{itemize}

O pioneiro e melhor exemplo de IaaS atualmente é o Amazon EC2\footnote{Elastic
Compute Cloud}, oferecido pela empresa Amazon. São oferecidas ao cliente
infraestruturas completas virtualizadas e todo o controle e gerenciamento pode
ser feito remotamente, utilizando uma API de serviços web. Outros exemplos são:
o SmartCloud\footnote{http://www.ibm.com/SmartCloud}, da empresa IBM, o
RightScale\footnote{http://www.rightscale.com} e
GoDaddy\footnote{http://www.godaddy.com/}.

\subsection{Tudo como Serviço (AaaS)} 
\label{computacaoemnuvem:classificacoes:servicos:XaaS}

Conforme aumentam os serviços prestados por provedores utilizando o paradigma da
computação em nuvem, diversos novos termos são criados. A sigla AaaS ou XaaS,
refere-se à frase em inglês \textit{Anything as a Service}, ou Tudo como
Serviço, onde a letra X pode ser substituída por diversas letras e assumir
vários significados diferentes. Esses modelos, apesar de utilizarem outras
nomenclaturas, acabam se derivando de modelos já consolidados e citados
anteriormente. Alguns exemplos:

\begin{itemize}
\setlength{\itemsep}{1pt}
\setlength{\parskip}{0pt}
\setlength{\parsep}{0pt}
\item Banco de Dados como Serviço: capacidade de utilizar os serviços de um
banco de dados hospedado remotamente;
\item Segurança como Serviço: fornecimento de serviços de segurança essenciais
remotamente via internet, como por exemplo, gerenciamento de identidades;
\item Teste como serviço: provimento de serviços de teste hospedados remotamente
para testar sistemas locais;
\item Informação como Serviço: capacidade de utilizar de qualquer tipo de
informação remota por meio de uma interface bem definida, como uma API;
\item Processo como Serviço, Gestão como Serviço, etc.;
\end{itemize}


\section{Classificação quanto ao Modelo de Implantação}
\label{computacaoemnuvem:classificacoes:implantacao}

Ambientes de computação em nuvem, além da categorização por modelos de serviço,
podem ser classificados em relação ao seu modelo de implantação, ou seja, sua
abrangência de público. Abaixo são citados os quatro modelos mais citados na
literatura.

\subsection{Nuvem Pública} 
\label{computacaoemnuvem:classificacoes:implantacao:publica}

A nuvem pública, ou nuvem externa, como alguns autores citam, descreve o
significado convencional da computação em nuvem, onde um prestador de serviços
disponibiliza recursos computacionais, tais como aplicativos e armazenamento
para o público em geral, através da Internet. Serviços de nuvem pública podem
ser livres ou oferecidos utilizando um modelo de pagamento baseado no uso
(\textit{pay-per-use}\footnote{Nesta modalidade o contratante solicita os
serviços e recursos de acordo com sua necessidade e disponibilidade e paga
apenas pelo que for utilizado.}), \citeonline{Smoot2011}. Empresa  como
Salesforce.com, Amazon EC2 e Flexiscale oferem esse tipo de serviço.

\subsection{Nuvem Privada} 
\label{computacaoemnuvem:classificacoes:implantacao:privada}

Em uma nuvem privada, uma infraestrutura é disponibilizada para uso exclusivo de
uma única organização que compreende vários consumidores (por exemplo, unidades
de negócios). Pode ser de propriedade, gerenciados e operados pela organização,
um terceiro, ou uma combinação deles, e podem existir dentro ou fora das
instalações da empresa\textsf{NIST}. Ferramentas como Eucalyptus e OpenNebula
permitem a implantação desse tipo de modelo.

\subsection{Nuvem Híbrida} 
\label{computacaoemnuvem:classificacoes:implantacao:hibrida}

Uma nuvem híbrida acontece quando recursos computacionais de nuvens públicas e
de nuvens privadas são utilizados. Uma empresa pode optar por usar um serviço de
nuvem pública para a utilização de recursos em geral, porém, pode armazenar suas
informações críticas de negócios em uma nuvem privada, dentro do seu próprio
domínio com a finalidade de aumentar a segurança dos dados. Em
\citeonline{Sotomayor2009}, uma nuvem privada, no entanto, pode dar suporte à
uma nuvem híbrida, através da complementação da capacidade da infraestrutura
local com a capacidade computacional de uma nuvem pública.

\subsection{Nuvem Comunitária} 
\label{computacaoemnuvem:classificacoes:implantacao:comunitaria}

Para o \textsf{NIST}, nesse modelo a infraestrutura da nuvem é compartilhada
por diversas organizações, dando suporte a uma comunidade específica, com
preocupações ou atividades em comum, podendo ser gerenciada pela própria
organização ou por terceiros e se localizar dentro ou fora dos limites da
organização.

\section{Padronizações para Computação em Nuvem}
\label{computacaoemnuvem:padronizacoes}

A indústria da computação em nuvem está nos estágios iniciais de implantação de
padrões, desde o armazenamento de rede à segurança. Hoje não há uma maneira
padrão para as empresas formatarem seus dados para que possam ser facilmente
movidos entre uma variedade de provedores de computação em nuvem. Se um cliente
têm os seus dados em uma nuvem AWS S3, da Amazon, por exemplo, não pode
necessariamente pegar esses dados e colocá-los em uma nuvem de outra provedora,
como Rackspace, usando as mesmas chamadas da API. Existe uma falta de padrões na
computação em nuvem. Sem padrões, a indústria acaba criando diversos sistemas
proprietários fechados e assim uma interoperabilidade entre fornecedores de
serviços se torna difícil. Como os clientes não querem ficar atrelados a um
único sistema ou fornecedor existe uma forte pressão da indústria para a criação
de padrões.

Uma série de organizações e grupos informais abordam esse problema. O site Cloud
Standards\footnote{http://www.cloud-standards.org} reúne através de projeto
colaborativo iniciativas de diversas organizações para a padronização em
computação em nuvem. Algumas dessas iniciativas são descritas na próxima seção.

\subsection{\textit{Open Cloud Computing Interface} (OCCI)}
\label{computacaoemnuvem:padronizacoes:occi}

O \textit{Open Cloud Computing Interface}, ou em português, Interface para
Computação em Nuvem Aberta é um conjunto de especificações de propósito geral,
baseadas em computação em nuvem, para interações com recursos de uma forma que é
explicitamente independente de fornecedor, e pode ser estendido para resolver uma ampla variedade de problemas em computação em nuvem.

O conjunto de especificações OCCI é um produto do \textit{Open Grid Forum}
(OGF), uma organização de desenvolvimento de padrões abertos na área de redes
distribuídas, computação e armazenamento, com ênfase em tecnologias de grande
escala em computação distribuída. O OGF desenvolve seus padrões através de um
processo aberto que reúne entradas e contribuições da comunidade e realiza um
refinamento através de revisões e comentários do público, a fim de produzir
normas, orientações e informações de valor para a comunidade.

O OCCI fornece um protocolo e uma API para todos os tipos de tarefas de
gerenciamento de nuvem. O trabalho foi iniciado originalmente para criar uma API
de gerenciamento remoto para serviços baseados em modelo IaaS, permitindo o
desenvolvimento de ferramentas interoperáveis para tarefas comuns, incluindo
dimensionamento, implantação e monitoramento. Desde então, evoluiu para uma API
flexível, com foco na integração, portabilidade, interoperabilidade e inovação,
oferecendo ainda um alto grau de extensibilidade. A versão atual da interface é
adequada para servir muitos outros modelos, além de IaaS, incluindo, por
exemplo, PaaS e SaaS.

\subsection{\textit{Open Cloud Consortium} (OCC)}
\label{computacaoemnuvem:padronizacoes:occ}

O Open Cloud Consortium, de acordo com o seu site principal, foi formado em
2008, e seus principais objetivos são:

\begin{itemize}
\item Suporte ao desenvolvimento de padrões para computação em nuvem e de
\textit{arcabouços} para interoperabilidade entre nuvens;
\item Desenvolvimento de \textit{benchmarks} para computação em nuvem;
\item Suporte às implementações de referências para computação em nuvem, de
preferência implementações de código aberto;
\item Gerenciamento de infraestrutura para computação em nuvem para suporte a
pesquisas científicas, como a \textit{Open Science Data Cloud}.
\end{itemize}
Essas tarefas estão divididas em grupos de trabalho que apoiam o interesse e as
atividades dos membros da OCC. Os grupos de trabalho atuais incluem:

\begin{description}
\setlength{\itemsep}{1pt}
\setlength{\parskip}{0pt}
\setlength{\parsep}{0pt}
\item[\textit{The Open Science Data Cloud (OSDC)} -] Grupo de trabalho que
administra e opera uma nuvem com grande quantidade de informações para dados
científicos. Entre os membros deste grupo de trabalho incluem o Yahoo, que
contribuiu com equipamentos para a prova de conceito do OSDC e a Cisco, que
fornece equipamento para conectar os vários centros OSDC distribuídos
geograficamente.

\item[\textit{Projeto Matsu} -] Este grupo de trabalho está desenvolvendo uma
nuvem que pode ajudar em momentos de desastres naturais, proporcionando uma
capacidade elástica para processar dados geoespaciais. Armazenamento baseado em
nuvem e serviços de computação estão disponíveis para o projeto e podem ser
usados, por exemplo, para auxiliar o processamento de imagens de modo que essas
imagens possam ser disponibilizadas para aqueles que fornecem assistência a
desastres.

\item[\textit{OCC Virtual Network Testbed} -] É uma vasta área distribuída de
testes para redes virtuais. O foco inicial é comparar e contrastar várias
tecnologias para criação e gerenciamento de redes virtuais.

\item[\textit{The Open Cloud Testbed} -] Este grupo utiliza equipamentos e
várias redes internacionais de pesquisa nos EUA para testar as diferentes
tecnologias para nuvens de amplo alcance. A participação nesse grupo de trabalho
é limitado aos membros OCC que contribuem com recursos computacionais, como rede
e poder de processamento.
\end{description}

\section{Infraestrutura para Computação em Nuvem}
\label{computacaoemnuvem:infraestrutura}
Atualmente, existem diversas soluções para implantação de um ambiente de
computação em nuvem na modalidade IaaS.