% ----------------------------------------------------------
% PARTE - preparação da pesquisa
% ----------------------------------------------------------

\part{Preparação da pesquisa}

% ---
% Capitulo de preparação da pesquisa
% ---

\chapter{Preparação da Pesquisa}

% ---
\section{Motivação}
% ---

Com a utilização de computação autonômica para gerar uma ontologia de uma base
de conhecimentos para uma  Cloud, damos um grande passo para implementação de
uma nuvem privada seguinda uma arquitetura específica.

\section{Objetivo Geral}
\label{introducao:objetivos}
\begin{itemize}
\setlength{\itemsep}{1pt}
\setlength{\parskip}{0pt}
\setlength{\parsep}{0pt}
\item Análise comparada das principais ferramentas de levantamento de topologia de
rede, elementos e meta-dados por meio da MIB SNMP.
\item Seleção de meta-dados relevantes nessas ferramentas e implementação do
 algoritmo de criação dinâmica da ontologia para uma Cloud.
\end{itemize}

\section{Objetivos Específicos}
\label{introducao:objetivosespecificos}
\begin{itemize}
\setlength{\itemsep}{1pt}
\setlength{\parskip}{0pt}
\setlength{\parsep}{0pt}
\item Apresentar os conceitos de tecnologias envolvidas na computação em nuvem;
\item Pesquisar ferramentas para implantação e levantamento da topologia de um
ambiente de computação em nuvem e compará-los,
\item Desenvolver um algoritmo de criação dinâmica de ontologia para a nuvem,
\item Testar o algoritmo desenvolvido através de um estudo de caso.
\end{itemize}

\section{Organização do Trabalho} 
\label{introducao:organizacao}

\begin{description}

\item[Capítulo 1 – Introdução -] Apresenta introdução e contextualização ao tema.  
\item[Capítulo 2 – Preparação da Pesquisa -] Apresenta a motivação, objetivo
geral e os objetivos específicos do trabalho.
\item[Capítulo 3 – Revisão da literatura -] Apresenta a definição das
tecnologias envolvidas, definição de ontologia, algoritmos criação dinâmica, são
apresentadas algumas tentativas de padronizações utilizadas. São mostradas e
comparadas ferramentas para o levantamento de meta-dados na MIB.
\item[Capítulo 4 – Resultados -] Nesse capítulo são apresentados os passos para
a implantação de uma nuvem. As melhores opções de ferramentas
para levantamento da MIB. Definição da ontologia e apresentação algoritmo
desenvolvido para criação da ontologia. Bem como, os resultados obtidos após a
criação da mesma.
\item[Capítulo 5 – Conclusão e Trabalhos Futuros -] Este capítulo encerra o
trabalho com algumas conclusões e considerações finais. São apresentadas algumas
perspectivas para trabalhos futuros.
\end{description}
