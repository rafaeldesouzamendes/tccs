\begin{resumo}

O cenário do monitoramento de redes está em constante mudança. Não faz muito
tempo a evolução dos Grides para as Nuvens gerou uma necessidade de adaptação
das já consagradas ferramentas de monitoramento. No intento de se alinhar com
esta evolução o LRG implementou a sua própria nuvem privada e, recentemente,
visa desenvolver uma arquitetura de monitoramento autonômica. Em trabalhos
anteriores o PCMONS foi criado e implementado em conjunto a adaptações ao Nagios
e OpenStack ou OpenNebula. Essa nova proposta é um completo redesenho da
arquitetura e acréscimo de novos paradigmas. Para tal, é preciso repensar quais
ferramentas de monitoramento utilizar, bem como, de que maneira as métricas de
uma Nuvem serão armazenadas para sua análise, quer seja em tempo real ou quer
seja em um espaço temporal mais alargado. Isto possibilita uma possível análise
da frequência de eventos na nuvem e, eventual tomada de decisão relativa aos
recursos disponibilizados na mesma. Tomar uma decisão na gestão de recursos
através da análise estatística é o princípio que norteia a auto-gestão da Nuvem
adaptando-se a mudanças imprevisíveis e escondendo a complexidade intrínseca dos
operadores e usuários. Até alcançar tal nível de complexidade na implementação
do projeto, é preciso particionar o problema e observar bem o cenário real das
nuvens, onde uma imensa massa de dados monitorados é gerado devido a quantidade
massiva de eventos em uma “cloud” de arquitetura complexa e com grande
quantidade de máquinas físicas e virtuais. Portanto, não estamos mais falando na
tecnologia tradicional de RDBMS, já que atualmente existem vários sistemas de
banco de dados adaptados ao paradigma Big Data. O foco deste trabalho é
exatamente fazer um estudo que vai direcionar ao melhor “schema” NoSQL do
sistema HBase para armazenar as métricas da nuvem do LRG, neste percurso é
analisado o porquê da escolha do HBase, qual as vantagens e desvantagens de um
certo modelo de Banco de Dados adotado, bem como, uma comparação das vantagens e
desvantagens do uso do NoSQL e não do tradicional SQL neste contexto.
Posteriormente, consoante a evolução das pesquisas do LRG e eventual
implementação de mais partes do problema, será possível consolidar o trabalho
com a implementação final da nuvem autônoma proposta. 
\\\\
\noindent
Palavras-chave: Computação Autonômica. Teoria da Decisão. Massa de Dados,
Computação em Nuvem, NoSQL, Modelagem de dados HBase.

\end{resumo}